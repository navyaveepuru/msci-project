\documentclass[11pt, oneside]{article}   	% use "amsart" instead of "article" for AMSLaTeX format
\usepackage{geometry}                		% See geometry.pdf to learn the layout options. There are lots.
\geometry{letterpaper}                   		% ... or a4paper or a5paper or ... 
%\geometry{landscape}                		% Activate for rotated page geometry
%\usepackage[parfill]{parskip}    		% Activate to begin paragraphs with an empty line
\usepackage{graphicx}				% Use pdf, png, jpg, or eps§ with pdflatex; use eps in DVI mode
								% TeX will automatically convert eps --> pdf in pdflatex
								
%\usepackage{{Gratzer-Color-Scheme}	% Active to color theorems red and lemmas blue
	
\usepackage{amssymb}


\title{Summary of Gunnell paper \\
\small Article reference: Gunnell, Y., Gallagher, K., Carter, A., Widdowson, M. and Hurford, A.J., 2003. Denudation history of the continental margin of western peninsular India since the early Mesozoic–reconciling apatite fission-track data with geomorphology. Earth and Planetary Science Letters, 215(1-2), pp.187-201.}
\date{}							% Activate to display a given date or no date

\begin{document}
\maketitle
\section{Overview}
The aim of the study is to extract the denudation history of the passive margin in the Western peninsula of India.
The post-rift denudation history is the controlling factor on the spatial and temporal extent of sediment loading offshore. Hence, understanding this will allow us to better understand passive margin models.
\section{Location}
\subsection{Passive margins; overview}
Mature passive margins represent an extensional environment in the transition from coastal to continental tectonic plates. They are no longer an active plate margin.
Extension is driven by plate boundary forces creating a zone under tension. Passive margins are the end member of this rifting process, and are therefore a site of sediment deposition. 
Onshore denudation causes significant sediment unloading, and along with isostatic rebound this allows us to understand the morphology of the landscape.
\subsection{Regional setting}
The study location is peninsular India, which has been subjected to major rifting events since the early Jurassic (ca 180Ma; Sorey, 1995).
It has since evolved into a mature passive margin, which is reflected in its current topography. Its morphological succession from the coast to the continental interior is as follows:
\begin{enumerate}
        \item A low lying coastal plateau with short, seaward-flowing rivers (Konkan-Kanara lowlands)
        \item A coast-parallel escarpment 0-70km inland (relief 0.6-2.2km). This is the continental-scale Western Ghats escarpment and is the main focus of the study. It forms the vast majority of the main drainage divide of peninsular India.
        \item An elevated inland plateau (Karnataka and Maharashtra uplands)
\end{enumerate}
Surface uplift in the Western Ghats escarpment is mostly due to denundational isostasy. The Northern part may include a surface uplift contribution from mantle processes, but this is geographically restricted as only the Northern end is associated with surface volcanics (late Cretaceous to early Tertiary continental flood basalts (CFB)), whereas the geology of the Southern end is primarily the Precambrian gneiss and greenstone basement (Dharwar craton). 
The Western Ghats is predominantly an erosionally formed landscape; although there is a seaward-dipping monocline in the continental flood basalts in the Northern end which may indicate either syn-rift rollover or post-rift localised CFB uplift, this structure has been breached relatively uniformly by receding drainage systems.
\section{Methods}
\subsection{Samples}
This study uses 92 apatite samples which were collected in total across major topographic variations in the study area, which lies from 10 to 16 degrees North and encompasses the lowlands to the West, the uplands to the East and the Western Ghats escarpment that marks the transition between the two. Figure 1 shows the locations of the samples, with the dashed line on the escarpment marking the transition from lowland to highland. Of the lithologies of the Western Ghats, the CFB of the Northern end and Archaean greenstone rocks in the higher summits limited the extent of apatite sampling for this study due to not containing enough apatite for analysis. Although the former prevented sampling further North, the denudation history of the higher elevations could still be found via extrapolation. 
\subsection{AFT explanation}
The resulting apatite crystals were 80 to 500 micrometers in size. After polishing and etching, low uranium muscovite was used as an external detector. After the irradiation of the samples, analysis can begin. [Talk about relationship between mean track length and AFT age]. Track length measurements are taken using a microscope for visibility, and calibrated against a stage micrometer. Finally, fission track ages are found using a zeta-calibration approach. 
\\Once fission track ages are found, discrete thermal events can be identified. A boomerang-shaped trend indicates the common scenario of differential cooling, where it is likely that the younger the sample is, the more cooling it experienced (and vice versa). The AFT age can approximate the age of the cooling event using the youngest samples, unless the mean track lengths are less than 13.5 micrometers. In this case, the AFT age instead provides an upper limit for the age of the cooling event (ie. it cannot be any younger than the AFT age provided). [why the 13.5 limit?]
\subsection{AFT data discussion}
A complex cooling history for the Western Ghats is revealed. The complexity is emphasised by the significant scatter in the relationship between AFT age and MTL, amd the relatively short MTL for a majority of samples (Figure 2). This suggess variable cooling took place, and thus the thermal history is more complex than the typical volcanic margin formation thermal sequence of a single cooling event related to a change in heat flow. 
\\The majority of samples have ages between 150 and 300 Ma and MTLS between 11.5 and 13 micrometers. 
As the elevation of the study area increases from West to East, so do the AFT ages. \\Relatively old ages are defined as greater than 220 $\pm$ 10Ma, and are found in the higher elevations (above 400m). The oldest ages are found in the highest elevations. Some relatively old ages are found in the low elevation coastal zone, although this is not an unusual observation for passive margins [cite; eg. SW Africa, SE Australia]. 
\\Relatively young ages are defined as less than 220 $\pm$ 10Ma. 
The youngest AFT ages are 55-75Ma and these have the longest found MTLs of more than 13.5 micrometers, suggesting that they record a relatively discrete cooling event. Their cooling history therefore cannot be fully explained by denudation alone. These youngest samples are close to the Northern CFB; given that the Deccan CFB episode has an age of 65Ma, they are likely correlated with a thermal event linked to the formation of the CFB. These samples are highly localised, since they are surrounded by samples at similar elevations with an AFT age three times greater. The regional nature of these very young samples therefore confirms that the heat source for this thermal event was not the Deccan source plume. It has been suggested that the heat source may have been reheating and then rapid cooling due to the burial of the Pre-cambrian basement by a lava pile which has since been eroded (taking into account the relatively low thermal conductivity of basalt, the lava pile must have been 2 to 4km thick to achieve total annealing). However, this explanation cannot fully justify the localisation of these ages. Instead, the paper suggests that these youngest ages are due to local resetting due to either local lava flows or late minor intrusions [cite], both of which are common along the coast but difficult to identify due to generalised deep lateritic weathering. If local lava flows are the culprit, these samples must have been close to the pre-eruption surface to feel the thermal effects. 
\\Young ages were also found much further from the Northern CFB, although these have an average MTL of less than 13 micrometers. Relatively young ages were found exclusively at low elevations (no age less than 150Ma is found above 400m).
\subsection{Quantitative history extraction}
Quantitative thermal and denudation history in this study were found using individual thermal modelling for each sample with a guided search and maximum likelihood criterion, and shared parameter space for temperature and time across all samples. This is the optimal method to fit observed track counts and length data. In order to simulate the temperature and time dependence of fission track annealing in apatite, an empirically derived algorithm devised by Laslett et al was used. 
\\However, this algorithm has a shortcoming; it appears to underestimate the amount of annealing at temperatures lower than 50 to 60 degrees celsius, so that samples at temperatures below this are falsely implied to have actually been at temperatures higher than this due to the fact they have more annealing. This causes a gap between the known surface sampling temperature and the unusually high temperature inferred by the Laslett et al. model from the observed amount of annealing. Over 1km of recent denudation would be required to bridge this gap, which did not occur.  
\\This shortcoming is explained by two observations. First, the assumption made during calibration  was that the uncertainty around unannealed (initial) track lengths (which were given a value of 16.3 micrometers based on induced track length measurements) in apatite was negligible. More recent work has shown that initial track lengths for induced tracks are around 5 percent greater. 
Second, geological samples which have remained under 40 degrees celsius have MTLs shorter than predicted by the algorithm. In samples where amount of annealing is assumed to be negligible (ie. samples used as age standards such as the Durango apatite), the MTL is around 10 percent lower (14.5 to 15 micrometers) than for induced fission tracks. This means that initial track length for geological, spontaneous fission tracks is lower than that for induced, laboratory fission tracks. 
This paper aims to address this shortcoming in the Laslett et al. algorithm by exploring new parameterisations. This is done by exploring the effect of using an initia lrack length of 14.5 micrometers instead of the conventional 16.3 micrometers. The aim of this empirical approach is to compare the results with independent chronostratigraphic data to assess the viability of this alteration. 
\\For each sample, the thermal history that fits the data best was converted to a denudation depth using Fourier's law for heat conduction: $ z = (T(z) - T_{s})*(k/Q) $
\\The parameters for the Dharwar craton, which is a stable cratonic shield, were set as follows:
\begin{itemize}
        \item[$z$] as depth
        \item[$ T_{s}$] as surface temperature. This is assumed to be a constant value of 20 degrees celsius. 
        \item[$k$] as average thermal conductivity of rock down to depth $z$. This is assumed to be 3.25 watts per meter per kelvin.
        \item[$Q$] as the surface heat flow. This is fairly low and homogeneous in the present day (35 to 40 meter-watts per meter squared, with temperature gradients of 11 to 15 degrees celsius). A spatially constant value of 40 meter-watts per meter squared for the present day is assumed. 
\end{itemize}
This results in a constant geothermal gradient of 12.3 degrees celcius per km.
\\The paper also took into account changes in:
\begin{itemize}
        \item[heat flow $Q$] (an exponential decay in heat flow since continental break up). Increasing $Q$ led to a decrease in denudation depth.
        \item[thermal conductivity $k$] (by $\pm 50 \%$). Increasing $k$ led to an increase in denudation depth.
\end{itemize}
Changing $Q$ and $k$ still maintained the overall denudation pattern (ie. the peaks were still peaks). Therefore, the denudation chronology is accurate even when taking into account the thermal parameters used. In particular, rift-related heat flow $Q$ tends to initially increase, and then decrease slowly with a time constant of $\approx$60Myr (ie. this is the time taken for the heat flow to reach 63.2\% of its initial value).
\\ 


\section{Denudation chronology}

\end{document}  