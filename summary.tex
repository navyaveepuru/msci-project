\documentclass[11pt, oneside]{article}   	% use "amsart" instead of "article" for AMSLaTeX format
\usepackage{geometry}                		% See geometry.pdf to learn the layout options. There are lots.
\geometry{letterpaper}                   		% ... or a4paper or a5paper or ... 
%\geometry{landscape}                		% Activate for rotated page geometry
%\usepackage[parfill]{parskip}    		% Activate to begin paragraphs with an empty line\usepackage{graphicx}				% Use pdf, png, jpg, or eps§ with pdflatex; use eps in DVI mode
								% TeX will automatically convert eps --> pdf in pdflatex
								
%\usepackage{{Gratzer-Color-Scheme}	% Active to color theorems red and lemmas blue
	
\usepackage{amssymb}

%SetFonts

%SetFonts


\title{Summary of Gunnell paper \\
\small Article reference: Gunnell, Y., Gallagher, K., Carter, A., Widdowson, M. and Hurford, A.J., 2003. Denudation history of the continental margin of western peninsular India since the early Mesozoic–reconciling apatite fission-track data with geomorphology. Earth and Planetary Science Letters, 215(1-2), pp.187-201.}
\date{}							% Activate to display a given date or no date

\begin{document}
\maketitle
\section{Overview}
The aim of the study is to extract the denudation history of the passive margin in the Western peninsula of India.
The post-rift denudation history is the controlling factor on the spatial and temporal extent of sediment loading offshore. Hence, understanding this will allow us to better understand passive margin models.
\section{Location}
\subsection{Passive margins; overview}
Mature passive margins represent an extensional environment in the transition from coastal to continental tectonic plates. They are no longer an active plate margin.
Extension is driven by plate boundary forces creating a zone under tension. Passive margins are the end member of this rifting process, and are therefore a site of sediment deposition. 
Onshore denudation causes significant sediment unloading, and along with isostatic rebound this allows us to understand the morphology of the landscape.
\subsection{Regional setting}
The study location is peninsular India, which has been subjected to major rifting events since the early Jurassic (ca 180Ma; Sorey, 1995).
It has since evolved into a mature passive margin, which is reflected in its current topography. Its morphological succession from the coast to the continental interior is as follows:
\begin{enumerate}
        \item A low lying coastal plateau with short, seaward-flowing rivers (Konkan-Kanara lowlands)
        \item A coast-parallel escarpment 0-70km inland (relief 0.6-2.2km). This is the continental-scale Western Ghats escarpment and is the focus of the study. It forms the vast majority of the main drainage divide of peninsular India.
        \item An elevated inland plateau (Karnataka and Maharashtra uplands)
\end{enumerate}
Surface uplift in the Western Ghats escarpment is mostly due to denundational isostasy. The Northern part may include a surface uplift contribution from mantle processes, but this is geographically restricted as only the Northern end is associated with surface volcanics, whereas the geology of the Southern end is Precambrian basement.

\section{Methods}
\subsection{AFT explanation}
\subsection{Past research}
subsection



\end{document}  